\documentclass[11pt, onecolumn]{article}
\newcommand{\myreferences}{C:/workspace/github/bibliography-jgr/bibliojgr}
\usepackage{graphicx}
\usepackage{subfigure}
\usepackage{amsmath}
\usepackage{booktabs}
\usepackage{longtable}
\usepackage{fancyhdr}
\pagestyle{myheadings}
\usepackage{float}
\usepackage{graphicx}
\usepackage{epstopdf}
\usepackage{textcomp}  %for degree symbol
\usepackage{natbib}
\usepackage{breakcites}
%\usepackage{apacite}
\usepackage[table]{xcolor}
\definecolor{lightgray}{gray}{0.9}
\usepackage{multirow} 
\usepackage[affil-it]{authblk}  %package for multiple authors
%\graphicspath{{C:/workspace/figures/}}
\graphicspath{{C:/workspace/figures/}}
\usepackage[utf8]{inputenc}
\usepackage[T1]{fontenc}
\usepackage{lmodern} % load a font with all the characters
\newcommand*{\addheight}[2][.5ex]{%
\raisebox{0pt}[\dimexpr\height+(#1)\relax]{#2}%
}
\begin{document}
%The end of the expert, dialogues in post democracy
\title{The technological phagocytosis of capitalism}

\author[1]{Jaime Gomez-Ramirez\thanks{Corresponding author \hspace{0.6cm} jaime.gomez-ramirez@sickkids.ca}}
\affil[1]{}
%\twocolumn[
%\begin{@twocolumnfalse}
\date{}
\maketitle

\begin{abstract}

2016 has been the year of Brexit and the rise and seize of power of Donald Trump. Both events were considered to be highly improbable, even implausible  according to the large majority of experts.
Whether this lack of accuracy in forecasting is a technical matter that  pollsters and statisticians need to take care of or the democratic manifestation of a fundamental malaise in the social fabric is in need of an answer.
Here we explore the driving forces behind what is conceived for some as the biggest crisis in capitalism in modern history. We argue that capitalism is dying of   success in commodifying goods and people. 

\end{abstract}

\section{Introduction}
\label{se:intro}

An stagnant global economy, increasing inequality, depreciation of wage income, job insecurity, those are some of the uncontested factors, for both right and left of the political spectrum, that are at the basis of the extraordinaire political outcomes we are living through.
But are these concerns just small fires that will be tamed with the appropriate political and/or economic agenda (depends who is asked) or is society heading towards a  socioeconomic reality entirely new.

Here we argue in favour of the last. We pursue a historical and multilateral investigation aiming at providing a rationale on how did we get into this situation. In the last part of the paper we discuss the potential impact on democracy and capitalism of new technologies, notably, digital currencies. 

\section{Historical account}
\label{se:hisacc}
From brettom woods to 1971 to now. 
commodification of money and everything else.
recycling surplus and deficits

\section{Here we stand, do we?}
\label{se:hisacc}
why the middle class is passe'.
why labour doesn't matter for capitalism, extract labour out of labourers, without having to deal with them,that is what uber does.
What is money and what it matters, is a good,a commodity, is debt? why its special status.
why now that money is cheaper than ever before in history there is a saving glut, rather than investing as economic theory would have predicted.

\section{Democracy and propaganda}
\label{se:hisacc}
why the minority rules rule and what it explains about the immense roi of political donors, persistence of koch brothers. christians won because they were fanatically stubborn, romans, the majority could not resist the minority. 
tax evasions laws designed with this in minds (veto)
%, see monti saw it as a competition probemand therefore 


\section{Don't listen to the experts}
\label{se:hisacc}
why science is not neutral and how it contributed to deplete the alternatives and shorten the narratives.
Economics mono - culture, all predict the same and all wrong.
Optimization and the economicus man, cold war engineering
who is stupid in the it is the economy stupid mantra.

\section{Technology killed capitalism}
\label{se:hisacc}
job force irrelevance, turing test passed
machines excel at doing loops, artists and profession with spark of creativity escape to the machine march.
commodification of human beings make them cheap and redundant

\section{Digital currencies}
\label{se:hisacc}
from extractive to distributed model.
digital currencies can deliver micro payments with no friction (no banks involve)

\end{document}

%%%%%%%%%%%%%%%%%%%%%%%%%%%%%%%%%%%%%%%%%%%%%%%%%%%%%%%%%%%%%%%%%%%%%%%%%%%%%
%%%%%%%%%%%%%%%%%%%%%%%%%%%%%%%%%%%%%%%%%%%%%%%%%%%%%%%%%%%%%%%%%%%%%%%%%%%%%
%%%%%%%%%%%%%%%%%%%%%%%%%%%%%%%%%%%%%%%%%%%%%%%%%%%%%%%%%%%%%%%%%%%%%%%%%%%%%
\section{Introduction}
\label{se:intro}



In 1997, during the semiannual monetary policy report delivered by the Chairman Alan Greenspan, he declared:

\begin{quote}
In 1991 at the boom of the recession, a survey of workers at large firms by ... indicated that 25 percent of workers feared being laid off. In 1996 despite the sharply lower unemployment rate and the tighter labour market, the same survey organization found that 46 percent were fearful of a job layoff.
The reluctance of workers to leave their jobs to seek other employment as he labour market tightened has provided further evidence of that concern, as has the tendency toward longer labour union contracts.
Thus, the willingness of workers in recent years to trade off smaller increases in wages for greater job security seems reasonably well documented. The unanswered question is why this insecurity persisted even as the labor market, by all objective measures, tightened considerably. One possibility may lie in the rapid evolution of technologies ...Continuing education is perceived increasingly necessary to retain a job.
...
At some point the trade off of subdued wage growth for job security has come to an end. In other words, the relatively modest wage gains we have experienced are a temporary rather than lasting phenomenon because there is a limit to the value of additional job security people are willing to acquire in exchange for lesser increases in living standards. ...the rate of wages will have to revert sooner or later to the rate of inflation.   
\end{quote}
%this is a contradiction on std economic theory and the frecast prove wrong, the living stdsincrese as much as inflaction, that is, 0 or -.


\section{Religare}
\label{se:religare}
Taleb's:
\begin{quote}
Purely monotheistic religious such as Protestant Christianity, Salafi Islam, or fundamentalist atheism accommodate literalist and mediocre minds that cannot handle ambiguity.)
\end{quote}
McDonald’s thrive because is a best worse-case divergence from expectations: a lower variance and lower mean. When there are few choices, McDonald’s appears to be a safe bet. Lingua franca languages can come from minority rules and that is a point that is not visible to linguists. 
“In Canada, when we say bilingual, it is English speaking and when we say “French speaking” it becomes bilingual.”
So all Islam did was out-stubborn Christianity, which itself won thanks to its own stubbornness. For, before Islam, the original spread of Christianity in the Roman empire can be largely seen due to the blinding intolerance of Christians, their unconditional, aggressive and proselyting recalcitrance. 
Christians were intolerant of Roman paganism. The “persecutions” of the Christians had vastly more to do with the intolerance of the Christians for the pantheon and local gods, than the reverse. 

Ray Dalio , %https://www.youtube.com/watch?v=U-OpDSP96Ro
All of the consensus already baked into the price, in order to make money in the market you need to see something that the consensus does not see, so you need to have an independent point of view.
While in other professions this was not the case, it is becoming more and more as the professions get more financialized. Professions that were not a zero sum game (you might not need to be the best doctor, just someone that knows ho to fix a leg) are indeed now following the same non-consensus approach, with purely survival gains.


%http://www.nobelprize.org/mediaplayer/index.php?id=1741
Thomas Sargent: Fiscal crisis often produces political revolutions, as in France 1789 and the US 1787.
%https://youtu.be/xiUuQSCR4h8?t=3h7m35s
Minsky? depression is not only people that have no jobs and decreasing standard of living but also smart people that are not given the time to think big thoughts. Who is going to spot a brilliant student and give him 5, 10 to yield a brilliant theory, in adverse economic conditions? Nobody will pay them for apparently doing nothing for 4,6 years.

%https://t.co/9xEuMfJEIz (Paul Mason)
In 1945, social democracy had to be reinvented across continental Europe, this time as both an overt and covert bulwark against Soviet influence. As it was born to courtail Sovietism it dies too when Sovietism died.
Karl Polanyi : capitalism consists of a “double movement” – the push for free markets and the pushback against them, to regulate them in the interests of society. This allow the liberal elite to justify its killing of the working class, social democracy was concerned no more with protecting it but with regulating capitalism for its own good.
There was a right wing version of neoliberalism, red in tooth and claw; and a progressive version – with its financial inclusion agenda, gay marriage, and meritocratic ethos in education, health care and social policy.


Corporate Tax avoidance promoted by the state (of Ireland)

The EC has a duplicity of views, it is seen as a uber free marketeer in social democratic countries (scandinavia, france) and at the same time in the uk and ireland as a bureucratic extractive force that is just on the way of the unique market.
But paradoxically the EC may be a force against too much globalization, for example in policying the irish facilitation of the tax loopholes that avoid taxation of the profits generated by the sale of apple products  in european soil.
But an interesting point is that should the tax be paid where the product is sold, what about where the product was designed and marketized?
Another issue here is that Apple’s war chest of profits has not been taxed (yet) because they have not (yet) repatriated it to the United States. A perfect example of a loophole (1986?  passed by the US Congress) allows that the the tax will be due when they take it home.
Importantly, what Apple does is tax  avoidance (blatant or otherwise) is by definition legal (Tax evasion is not).
The double Irish with a Dutch sandwich is a tax avoidance technique employed by certain large corporations, involving the use of a combination of Irish and Dutch subsidiary companies to shift profits to low or no tax jurisdictions. 
%http://www.investopedia.com/terms/d/double-irish-with-a-dutch-sandwich.asp
Double Irish With A Dutch Sandwich : send profit to an irish subsidiary, then to a dutch and then back to a second ireland company headquarter in a tax heaven.

Mario Monti: was the first one to really suggest that tax avoidance is a competition issue (He should know, coming from Italy). And from that, taxation is subject to the unanimity rule in the Council, which means Ireland will simply veto everything, and therefore nothing will happen.


Economists and pro-trade activists called it “The Great Moderation.” (Bernanke) The world was going to be more global, richer, happier. This was a huge corporate propaganda campaign. 

%http://theantiglobalist.com/global-trumpism-why-trumps-victory-was-30-years-in-the-making-and-why-it-wont-stop-here/  (Blyth)
The era of neoliberalism is over. The era of neo-nationalism has just begun.
The problem with doing so, over time, is that targeting any variable (unemployment rate) long enough undermines the value of the variable itself, a phenomenon known as Goodhart’s law.  Michal Kalecki: once you target and sustain full employment over time (tight job market), it basically becomes costless for labor to move from job to job. Wages (employers to retain employees) in such a world will have to continually rise to hold onto labor, and the only way business can accommodate that is to push up prices. This pushes inflation because prices and wages push each other up. In the 70s the end of Keynesianism with too much inflation (oil shock, end of the gold-dollar peg monetary system etc.) the system undermined itself, as both Goodhart and Kalecki predicted. 

In Bretton Wood era 45-70, it was a great time to be a debtor, it was a lousy time to be a creditor. Inflation acts as a tax on the returns on investment and lending. Unsurprisingly, in response, employers and creditors mobilized and funded a market-friendly revolution where the goal of full employment was jettisoned for a new target—price stability, aka inflation, to restore the value of debt and discipline labor through unemployment. And it worked. 

The new order was called neoliberalism. Over the next thirty years the world was transformed from a debtor’s paradise into a creditor’s paradise where capital’s share of national income rose to an all-time high as labor’s share fell as wages stagnated. 
%although interest rate is very low and low inflation or precisely for that, the creditors are in the drivers seat, because the debtor is not killed (low interest rates) but ant pay it back easily (no inflation, not stagnated wages). A dominance creditor-debtor that extends over time (comapred this with higher interest rates but short periods)

Wages collapsed due to the twin shocks of restrictive legislation and the globalization of production. % see that the estate came always to the rescue of neoliberalism, passings las to cheapen labour.

But Goodhart’s law never went away. Just as targeting full employment undermined itself, so did making inflation the policy target.
Phase I (post ar Bretom Woods) 45-70 target was unemployment, 70-2008 (Pos Nixon's shock) the target was inflation. 
This was so successful that since 
2008- the world’s major central banks have dumped at least $12$ trillion dollars into the global economy and there is barely any inflation anywhere! 
In targeting inflation we created a world that is deflationary (we did the job too good) but when you have a lot of debt and the word is deflationary you cannot possibly service your debt, unless you stop spending, which again makes it more deflationary, no spending companies reduce prices ...
What we see is a reversal of power between creditors and debtors as the anti-inflationary regime of the past 30 years undermines itself, the middle class were the debtors, but now as debtors in a deflationary world are powerless. %they have the votes, but remember minority rules.

In this world, yields compress and creditors fret about their earnings, demanding repayment of debt at all costs. Macro-economically, this makes the situation worse: the debtors can’t pay—but politically, and this is crucial—it empowers debtors since they can’t pay, won’t pay, and still have the right to vote.

Philip Mirowski
%http://www.nakedcapitalism.com/2016/05/philip-mirowski-this-is-water-or-is-it-the-neoliberal-thought-collective.html
(touched a nerve with some economists)
(left gives too much credit on neoliberlism verging dangerously towards conspiracy theories) 
%technology will accomplish what the left never could, overthrow capitalism, or is capitalism organic and wil mutate into something else

Perhaps neoliberalism is simply what we get when the boss class exercises power over the state.
Another HisMat is a conviction that everything ‘really’ just boils down to economics/statistics (Gelman blog that "explains" why polls got it wrong), and that one can ignore anything else as just ‘fluff’. 

The rise of the Neoliberal Thought Collective cannot possibly be understood narrowly as an offshoot of ‘economics’ as such; rather, it is a general philosophy of politics and the meaning of life.
Yet I believe it is precisely this disdain for political ideas and simplistic appeals to ‘capitalism’ that has crippled the Left in its search for something substantial with which to challenge the palpable dominance of neoliberal discourse in everyday life, as well as in political mobilizations.
Freedom is not ‘metaphysical’ or Kantian freedom where individual autonomy dictates reason governs emotional or intellectual impulses. Hayek’s psychology suggests hardly anyone is capable of exercising those faculties: freedom is not a power or capacity.
Hayek defines liberty in the negative "personal freedom is defined as an absence from “coercion by the arbitrary will of another”".
 the establishment of laissez faire (far from doing away with the need for control, regulation and intervention, enormously increased their range. Administrators had to be constantly on the watch to ensure the free working of the system.)
 
 
Neoclassical economics as an evolutionary organism:
 its origins in 19th century physics (Walras...Samuelson physics envy); 
 it’s curious revulsion from indeterminist currents in the natural sciences(??);
 its mutations on both the left and the right; its hybridization with the military and operations research; (Bellman)
 (See economics as engineering) %https://www.ineteconomics.org/perspectives/blog/economics-as-engineering-iii-carnegie-stories
 its symbiosis with the computer; and lately, ()
 its conversion to a theory of “information” and an engineering set of ambitions


NeoLib, on the one hand, pretends to be the continuator of a heritage dating back Adam Smith) and on the other hand is an amalgamation of the Austrians, the Ordoliberals \footnote{German variant of social liberalism that emphasizes the need for the state to ensure that the free market produces results close to its theoretical potential. an offshoot of classical liberalism that sprouted during the Nazi period, when dissidents around Walter Eucken, an economist in Freiburg, dreamed of a better economic system. They reacted against the planned economies of Nazi Germany and the Soviet Union. But they also rejected both pure laissez-faire and Keynesian demand management.}, and the American Chicago School. 

The real key to understanding modern economics is the absorption of the central tenet of markets as superior information processors within the heartland of cutting-edge microeconomics.
To propose Keynesian solutions to the crises of the 21st century is to ignore three obvious problems. It is hard to mobilise people around old ideas; the flaws exposed in the 70s have not gone away; and, most importantly, they have nothing to say about our gravest predicament: the environmental crisis.
The fact that NTC virulently atacks Keynesianism, doesn't make Kynesianism any good.
Keynes was the exemplary Classical Liberal of the interwar period.This political thinking has been cannibalized by neoliberalism.
Endless denunciations of “capitalism” do little more than evoke a thoroughly superseded HisMat, which lacks all relevance to the modern situation


Andy Haldane: whose recovery?
%http://www.bankofengland.co.uk/publications/Pages/speeches/2016/916.aspx
UK’s “recovery is real in macroeconomic data: GDP is 7pc higher, employment 6pc higher and wealth over 30pc higher than in 2008. So why has that recovery not been felt, at least by some? 

Aggregate activity measures are sometimes a poor
proxy for the average person’s income. GDP per capita has risen significantly more slowly than the aggregate GDP since 2009. Indeed, GDP per head today is only around 1pc above its pre-crisis peak. GDP measures income from all UK-based activities. But not all of that income flows to UK citizens. Some
flows abroad to foreign owners of UK assets and some income is earned domestically on UK investments abroad.

Job insecurity would tend to lower the likelihood of people moving jobs.  job-to-job flows began rising slowly after 2010 but remain at levels well below their pre-crisis levels. Meanwhile, surveys suggest around a quarter of workers currently are afraid of losing their jobs, the highest level in several decades.
Nominal earnings today are around 7pc higher than in 2009. But in real terms earnings are still around 5pc below this peak.  

%http://www.bankofengland.co.uk/publications/Documents/speeches/2016/speech937.pdf
Although this crisis in economics is a threat for some, for others it is an opportunity – an opportunity to make a great leap forward.
So even physics, for some the theoretical pinnacle of the natural sciences, is these days far from being a precise science (the LHC is no more than a huge monte carlo machine ...Modern physics (neutrino) come in the sort of huge fishing expeditions . industrial-scale searches for needles in haystacks, often using large-scale computational techniques.) Remind that in general, a system comprising three bodies undergoing motion via a force cannot be solved analytically.
The deductive approch in science dates back to Popper's "The Logic of Scientific Discovery", where he champions the idea that knowledge advances  setting a set of axioms and from those were deduced a set of logical propositions or hypotheses. And then, and only then, were these hypotheses taken to the data to be validated or falsified. 
(To's IIT embodies exactly this. )

Newtonian physics was built on the same principle of internal consistency within systems, energy always preserved. Also dynamic optimisation techniques used initially by Walras.
It does not escape to anybody that much of mainstream macro-economics and finance has essentially followed this intellectual lead. 
The assumptions or axions :  It typically
starts with a set of assumptions or axioms are the preferences of consumers and the technology facing firms.  From those assumptions are derived equations of motion for the behavior of consumers and firms (Euler equations). These models, so derived, predictably exhibit the same damped harmonic motion as Newton’s pendulum. 

The dominant approach over recent decades to modeling the
macro-economy has probably been the Dynamic Stochastic General Equilibrium (DSGE) model.  plain-vanilla form, this comprises a set of representative, optimising households and firms. 
This model gives rise to an equilibrium which is unique and stationary, and dynamics around that equilibrium which are regular and oscillatory.
But this economics and finance well-defined foundations. Lucas said d “beware of economists bearing free parameters”. He was right. A theory of everything is a theory of nothing.
Another advantage is the scalability, on the assumption agents’ behaviour is representative – it broadly
mirrors the average person’s – these models of micro-level behaviour can be simply-summed to replicate macro-economic behaviour. Agent s and firms and their aggregations are fractal: The individual is, in effect, a shrunken replica of the economy as a whole. Thus, the macro-model are said to be micro-funded, that is, constructed bottom-up from
optimising, micro-economic foundations.

Crucially, If the assumptions underlying these models are valid,
then the behavioural rules from which they are derived will be unaffected by changes in the prevailing policy regime. This is why the policy sphere embraced so effusively the DGSE model. It is immune to the Lucas' critique \footnote{'Lucas critique' is a criticism of econometric policy evaluation procedures that fail to recognize that optimal decision rules of economic agents vary systematically with changes in policy.}. But it is not to the Goodhart paradox! (see how wrong the models were in forecasting UK growth, different models from institutions are clustered around a 1pc difference, the methodological mono-culture produced made them forecast the same, and wrong)

Of the interdisciplinary "tale" only economists seems shamelesly wrong and also truly honest in their responses (responses by professionals in different social sciences to the proposition “inter-disciplinary
knowledge is better than knowledge obtained by a single discipline” (Fourcade, Ollion and Algan (2015)). A majority, often a large majority, of other social sciences agree with this statement. Only in economics do a majority disagree.) I presume that for natural science (excluded maths) the majority in favour of int-dsc would be even larger than the 60-80pc seen in social science (except economics)
Nature journal, calculated this: published interdisciplinary indices, measuring the number of references made to outside disciplines by a subject area and the number of citations made outside the discipline to that subject area. Over the period 1950-2014, economics sat in the bottom left-hand quadrant 

Turner has a more accurate and sophisticated view of banks than for example Dalio who thinks banks are places where people gather and bring money so that banks can lend it to someone else. Turner says:
"Read almost any economics or finance textbook, and it will describe how banks take money from savers and lend it to business borrowers, allocating money among alternative capital investment projects. But as a description of what banks do in modern economies, this is dangerous fictitious for two reasons. First, because banks do not intermediate already existing money, but create credit, money, and purchasing power which did not previously exist.  And second, because the vast majority of bank lending in advanced economies does not support new business investment but instead funds either increased consumption or the purchase of already existing assets, in particular real estate and the urban land on which it sits."
%Thus, the bottom line banks create money and do not invest in productive


Douglas Rushkoff
%https://www.ft.com/content/a4e7cda0-deda-11e5-b072-006d8d362ba3
Investors demand that companies constantly expand their market share and earnings, even though we know trees do not grow to the sky. The digital revolution has only sharpened these destructive impulses, writes Rushkoff. Yet, thinking like a computer coder, he argues that the economic system and our corporate priorities can be reprogrammed.
He claims that we can reprogram from going from an extractive to a distributive model (Amazon owned by its sellers, Uber owned by its drivers, and Facebook is owned by the people who create its content.) 
This is the same that Jaron Lanier said, the difference may be that D.R. embraces Blockchain, the BC can effectively redistribute the cognitive capital input by FB users. See also Varouf. with his plain of digital codes for making payments.

%http://www.forbes.com/sites/stevenrosenbaum/2016/03/01/douglas-rushkoff-rocks-the-google-bus/#33e8002a744b
The digital economy is a house of cards built on fictional growth metrics that drive companies to raise money, undercut human workers, sell on the public markets and then – almost inevitably – collapse under the weight of public market demands. (YS note the crisis of the IPO). 
Reject platform monopolies like Uber in favor of distributed, worker-owned co-ops, orchestrated through collective authentication systems like bitcoin and blockchains.

“Think of it less like a war, where total victory is the only option, and a bit more like peace, water eat objective is to find a way to keep it going.” This view is the only sane one in (physical) growth were infinitum growth is possible (and production upper bounded). 
Recognize contributions of land and labor as important as capital, and develop business ecosystems that invest in the local economies on which they ultimately depend.

Adair Turner
%http://privatedebtproject.org/view-articles.php?An-Interview-With-Lord-Adair-Turner-6
The fundamental problem is that modern financial systems left to themselves inevitably create debt in excessive quantities, and in particular debt that does not fund new capital investment but rather the purchase of already existing assets, above all real estate. In the decade running up to the 2007-08 crisis, private credit grew rapidly in almost all advanced economies:  in the United States at 9pc per year, in the United Kingdom at 10pc per year, in Spain at 16pc per year (YS nadie en el banco de espana vio eso??).
In most it grew faster than nominal GDP: as a result, private leverage—the ratio of private credit to GDP—significantly increased.
At the core of financial stability in modern economies, I would argue, lies the intersection between the infinite capacity of banks to create new credit, money, and purchasing power, and the scarce supply of irreproducible urban land.  


Kahneman

Michael Lewis about Tversky foundations of measurement: the whole enterprise had a tree-fell-in-the-woods quality to it. How important could the sound it may be, if no one was able to hear it?

A baseball taeam Using the same methods, they won it again in 2007 and 2013. But in 2016, after three disappointing seasons, they announced that they were moving away from the data-based approach and back to one where they relied upon the judgment of baseball experts. (“We have perhaps overly relied on numbers,” said owner John Henry.)

Engineering (Experts neutrality)
%https://www.ineteconomics.org/perspectives/blog/economics-as-engineering-iii-carnegie-stories
Lucas, Prescott and their colleagues built upon the Cold War “modeling strategies” previously devised by economists and engineers under the supervision of the military.
demands of military patrons in the context of the Cold War turned some branches of mathematics into a “science of economizing.” (production planning and warfare models simple enough to be analitically understood).
Three pillars of New Classical macrodynamics: 
computational and practical constraints(Bellman’s dynamic programming techniques)
Take for example Thomas Sargent who thinks that bounded rationality means optimization under constraints ("the more contraints have the people in the economists' models the smarter they are because the models are more demanding mathematically and econometrically")
%https://books.google.es/books?id=FR5Cee2IqkcC&pg=PA391&lpg=PA391&dq=thomas+sargent+constraint+agent&source=bl&ots=DbPi4s5nhM&sig=935sHCV3NQIpZPysnP4uLUm2awY&hl=en&sa=X&ved=0ahUKEwj3ho7wlrfQAhXB5xoKHeNtA_wQ6AEINjAE#v=onepage&q=thomas%20sargent%20constraint%20agent&f=false
In optimization under constraint the agents are one step beyond the gods  (created in the image of the econometricians -the gods) .
Bellamn's dynamic programming is easier than Pontriagyn calculus of variations (N-E equations find the curve of shortest length connecting two points. If there are no constraints, the solution is obviously a straight line between the points. However, if the curve is constrained to lie on a surface in space, then the solution is less obvious, and possibly many solutions may exist.geodesic) and all macroeconomic model is based in optimization under constraint ) Simon’s certainty-equivalence theorem and Muth’s notion of rational expectations, 

%https://www.ineteconomics.org/perspectives/videos/is-technology-killing-capitalism
Eric Weinstein (technology is killing capitalism)
millions of small fire or do we need some central explanation?
Recovery without progress in wages, middle class loosing ground, first home propensity decreases.
Was market capitalism an accident of 20th century and the conditions, requirements, constraints that were in place then, and are not there anymore invalidates market capitalism? 
Capt had socialism as its intellectual rival or dancing partner. The conditions have changed, not though ideology but through technology.
Software spends far more of its time in a loop. Crafts professions are based on the loop, the repetitive pattern to the pursue of expertise (coders, pilots,surgeons) so do work (loop) to acquire some expertise and feed our families.
A radiologist who has been trained for many years in order to make tricky diagnosis potentially will be obviated by a "deep learning" algorithm and this leads us to a very unconfortable paradigm.
Software is threatening not only low value repetitions but also high value repetition behavior.
The only place where humans have some edge over machines, creativity spark, singular act, creation. 
Imagine a world in which crypto currency have endogenous central bank algorithms built in so when recession hits the increment the money supply and to contract later on without having political agendas of people at the helm.
%Bitcoin isn't a debt but an asset). Bitcoins only deflate in value when the Bitcoin Economy is growing.
%A deflationary spiral occurs when there is an incentive to hoard because of declining prices, which results in even less available currency on the market, further perpetuating declining prices.
%https://www.youtube.com/watch?v=sirXAfpIrao


%Adair turner
DSGE assumes that policy can affect the rate of inflation and the volatility of output.
In this new deflationary world there are questions in need of an answer, for example, what sustainability means in a world of 0pc interest rate, can economies function well with -i.r. 
If ir are 0 or close to 0, then bonds and money become actually the same asset class (I rather stick to cash than buying a stock at maturity x that gives me 0pc yiend \footnote{The 30-year Treasury will generally pay a higher interest rate than shorter Treasuries to compensate for the additional risks inherent in the longer maturity).
Macroeconomics std model only care about the std volatility of output ...
Despite very low ir we still have very low growth and lower target inflation. 
Sec stag  hypothesis (alvin hanson): supply side (robert gordon) demand side.


DATA SMART

In the cover the book Data Smart by John Foreman there is an striking quote from one Patrick Crosby (the CEO of OhCupod) "Data Smart make modern statistics methods and algorithms understable and easy to implement. Slogging through textbooks and academic papers is no longer required! (The author works at Chimpmail)

Data science is synonymous with or related to terms like business analytics, operations research, business intelligence, competitive intelligence, data analysis and modeling, and knowledge extraction (also called knowledge discovery in databases or KDD). It's just a new spin on something that people have been doing for a long time.
Definition: Data science is the transformation of data using mathematics and statistics into valuable insights, decisions, and products.

Varoufakis (Politics over purely technological solutions)

Economics is not a natural science ad even worse is not accumulative, we seem to forget "truths" established from previous generations.
In order to say anything sensible about capitalism and democracy we need to go back to the basics, or follow a first principles approach.
What is capitalism (free markets?the invisible hand?)

Neo liberal agenda has undermine the narrative of the liberal class, they have no scrupules in getting rid of democracy , that is, popular will, as an example, Greece, new left wing government did not stand a chance to restructure the debt, markets overrule democracy.
The neo liberal apparatus has stopped repeating the mantra of democracy goodness for an "efficiency" discourse.
Plato was against democracy and Aristotle famously claimed that is the government of the poor and the free. This is the sentiment today, racial conflict is that, what if the poor, who let us not forget have the votes, unite and attack my priviledges and my social condition.
If power lies with the poor, as it does, this is simple arithmetics.

Scale: in Athens, scale allowed directed decision making 
%isegoria ‎(uncountable): equality of all in freedom of speech
Magna carta is a contract of the barons to limit the power of the  kings over the barons' plebes peasants. It is a social contract between the lords and the monarch.
Then the merchants who had no political power first.
And then the proletariat after the enclosure laws \footnote{enclosed open fields and common land in the country, creating legal property rights to land that was previously considered common}, began to organize.
And last stage the state emerges as a mechanism to avoid conflict among the incumbent, reduce class tension.
We need to study this evolutionary process for that we need to be aware of the classics of economic theory (Smith, Ricardo, Marx ... Schumpeter)

One key aspect is COMODIFICATION, explains why do we have an apolitical, a historical mathematized economic theory.
The march of commodification aka exchange vale, against experiential or use value, the inexorable march of the commodity that takes over all goods to commodify them (price them).

Kenneth Arrow's impossibility theorem has implications:
1. no state can be justified. this a sequitor from what the theorem demonstrates: there is no a well defined notion of common well). If the state cannot address the common well, why do we need the state in the first place? -Hayekians and libertarians agree with this point.
Can it work capitalism without state, Varoufakis says NO "it is like christianity without hell". But it is because they are complementary or opposites, is the state, state-ism what justifies capitalism?

Capitalism would not existed without estate brutish force, for example, Enclosure Act which created the proletariat in the UK, forcing people to leave the land and live in the city.
Open fields were commodified, commodification is the force that drives capitalism.

2. There is no solution to the problem of what we want even having perfect information about what we desire.
This is the problem of utility function, what is utility the desirability? but then, based on Arrow's, it does not translate in decisions.
This is indeed a vindication for democracy, because democracy is dialectic. A dialectic process is between two parts but more importantly is indeterminate (for Hegel, dialectical is negatively rational or moment of instability, “self-sublation, both cancel and preserve).

The imperative of capitalism is that it has an iron law that consists in commodify, cheap labour...take over labour. Ideally, capitalism would like to  take labour out of labourers to do not have to deal with them (perfect commodification). But the fact that this (perfect commodification) cannot be accomplished, it is precisely what gives value to commodities (CAN IT? Weinstein view in technology killed capitalism is that, labour can be effectively taken from labourers).
YS: Perfect commodities are deprived of value, price natural tendency to go down?
The better capitalism is at commodifying people the more social tension, the more in crisis is the estate, which it was there in the first place to facilitate capitalism and then to control from its own immolation (is ISIS a run away reivindication of the estate?)

Commodification is a process inherently unstable, we can't expect to find an equilibrium.
Historically, see the first constitution the US, the main concerned of the funding fathers was since its inception was to remove the demos from democracy or how to make sure, through representative democracy that the plebes are kept away from the levers of political power.
But the problem of legitimacy of political power is "resolve" through manufacturing consent.

The golden rule of the insider is that you never speak out about other insiders, outside.

The femminization of procession bring down the remuneration (see stats) min 50, when women grew in numbers in profession, wages went down (that is why first tweet ever of warrem buffet (Read my new essay on why women are key to America's prosperity: http://cnnmon.ie/18eXfik .) if we need more women in the work force and the lean in campaign).


Nixon's shock, passage first 1st postwar phase (Brettom Woods) to the second, suddenly the political decision were moved to macro bodies (EU).
Bretton Woods understood that unless there were politically motivated recycling of surpluses and deficits, at the global level, there was impossible to keep stable interest rates and a reasonable stable.
This was designed by White (against Keynes who wanted a global controller but White prevailed got the Americans to control the deficits-surpluses), not also that bankers were not invited to Bretom woods.

When the US stopped being the surplus-er and became deficit-er, the world geopolitics shacked, how the US will keep its hegemony now that it is not the surplus-er. The US did indeed increased hegemony as they increased their deficitarian condition!

So in 45 there was Europe that need to be rebuilt, the US was the surpluser and this went to Europe to be rebuilt, ths keep the economy, globally,growing and stable, one debt and other consumption.When europe was rich and rebuilt, they started exporting themselves, the US became and importer, which sucks, so they became the surplus (EU and China, Japan) of the world, how, not through infra estructure or other forms of deficit spending which built the social well fare in Europe but via Wall street, the surpluses from the non US was sent to the US via Wall Street. This is in essence how finantialization happened, capital flows going to WS.
For this new state of affairs, a new form of consent must be established and delivered to the masses, call it EMH, International Finantial macroeconomics.
This collapse out of its hubris in 2008, and since then we have not recovered we still in limbo, the recycling mechanism is severely broken down.

The IT will do what the left has failed to do, OVERTHROW capitalism (Weinstein is right!). The moment machines pass the Turing test, 3D printers as public utilities...we dont have a Schumpeterian creative destruction but economies of scale and corporate capitalism collapses.


Why europe let Greece in? they were the perfect debtors for French and German banks, greece had very low level of debt.
(min 1:43 greatdiscussion of the right of the french to speak, none, since the germans and only then have the get out of europe card, because they are the suplus-ers, france is just another deficit-er. Thats why the eurobonds will never happened the germans will lose their get out card, BUT how this relates with the "votes" that Dalio says the deficit-ers have?)

%https://www.youtube.com/watch?v=P2Zpkz7lK-s
Goods are commodities, it is exchanged. Money is a good and a commodity, it has a price and can be traded.
What is the price of money? the rate of interest.
Investment is when the yield of the asset we choose is larger than the money interest rate (no risk), if the money is expensive (high interest rate, will keep it, saving it) if the money is cheap I will tend to give it to someone else (invest it).
After 2008, something deeply worrisome happened because money is the cheapest in history, and we don't give it to anybody else, but we prefer to save. Money is a good with a negative price!
Because there are pesimistic expectations.

The recycling process of investing and savings is broken as well. Savings exceeds investing by 7 trillions (min15) never before in the history of capitalism we had this kind of gap. Normally when,as now Savings >> Investing, price of money (ir) must drop, so people invest and rebalance, incredibly, now having -ir the gap is not even not going away but increased.

This is animal spirits, if the Fed does not increase the ir (keep it cheap) the logic will say to invest it, but the human psyche thinks, if Janet doesnt hike the ir is because things are bad out there, therefore feels its guts rather than "economic logic" and keep over saving. So money is not as other commodities, for example take food, I have potatoes and nobody want them, if I want to clear up my stock just reduce the price. For money is not that simple, money price is psychological, the more I cheapen the price the more the other think it is prefereable to keep it,this way of thinking doesnt make sense for other commodities, imagine that potatoes are cheaper and cheaper and therefore I keep them. Maybe it is because potates are traded with money and money is a second order commodity, that is, the decision of save or invest is always about future flow cash, while potatoes perish or are less liquid, when i sell potatoes i dont expect to buy them back, as i do with financial assets.

What is so special about money? liquidity? is as good that everybody wants? its monopolisitc status?or it is that it is debt itself, not a commodity, banks know this well, their loans are assets not liabilities, the liabilities are the deposits.
So money is debt at fixed interest debt. In this optic we ca understand the present moment, a saver is someone that lends money to herself while an investor is someone that lends to someone else, when we feel or know that everybody is doing the same, lending to oneself we all do the same.
It goes without saying that this also holds for the debtor side, because capitalism and democracy and finances are  dialectic.
So the bottom line is the economic or financial decision making what really matters is the other's people mind problem!

Wages can't go up while the savings glut persist! In the US from the independence to 1973 wages have been increasing, now for the first time the estimates are that next generations will be paid less than the precedent! The preassure is on cheapening labour and other commodities, it is a deflationary world. QE cant be the pessimism which is what is pushing the deflationary forces.
Margaret Thatcher got it right, "who controls interest rate, monetary policies in Europe", controls the politics in Europe. 
How this stands with digital currencies? %YS

%https://www.youtube.com/watch?v=sirXAfpIrao&t=1094s
%Summers and Turner
ir problem is not a 2008 phenomenon is an after 71 one. it is not an overhang caused by the crisis. 
Demand based problem not supply based (robert gordon), lack of demand creates its own lack of supply.
Reducing ir also has the effect of pulling forward demand, you buy the has now rather than next year , then for next year you have already eaten up the demand if ir dont decline in the next year.
Secular stagnation is a curable malady, supply side is worse because it entails a limitation in the capacity of producing staff, but here we are in having no interest in consuming staff
 


Bitcoin problem: it will lead to stagnation since it is capped and you cant increase the money supply (it was the money supply increase and deficit spending what saved US in the thirties to join fascism).

%https://ftalphaville.ft.com/2015/05/27/2130503/summers-and-swiss-bitcoin-hoards/
Bitcoin as it stands is a system that pulls critical resources and capital away from producers and infrastructure developers and passes them over to predators or undeserving spendthrifts, often in a way that allows them to get something for nothing or which facilitates value hoarding.
Teach the bitcoin faithful the importance of investing rather than hoarding money?

Unlike the dollar money system, which draws value from the bonds (Kaminska here with candor calls bonds to debt) it creates between parties, Bitcoin is designed to count as an asset in its own right. 
In that sense, bitcoin represents a particularly unsociable type of money system. It is no-one’s liability. And that means no-one guarantees a bitcoin’s minimum worth.
 
YS: this may be true but she ignores the asymmetry of debt with is very evident in investment, it is essentially asymmetric, being wrong short is worse than being wrong long (see financial coursera u.geneve michel), the downside is faster than the upside. This bilateral self compensatory nature of money (investing)is dubious.
fiat money is the way the rich stay rich, before there were many moneys but then the powerful force to use only their money and to have it you have to borrow and pay back at interest (indentured servitude: in new england colonies (one-half to two-thirds of the immigrants who came to the American colonies arrived as indentured servants. Servants typically worked four to seven years in exchange for passage, room, board, lodging and freedom dues)).
En plus, they put a clock in money, that is why time is money (B. Franklin quote is thought to mean to gotta be productive using your time is desingenious).
Is the digital economy a way to go back to the 11, 12th of many moneys.
The visigothes in Spain "Con Chindasvinto el retrato del heredero pasó al anverso, apareciendo las dos cabezas de perfil y unidas, en un intento de mitigar los problemas de sucesión derivados del sistema electivo de la monarquía visigoda" 
Currency is about power not trade.
%Parias (Del b. lat. pariāre, igualar una cuenta, pagar), tasas, impuestos o tributos que los reyes musulmanes de España pagaban a los reyes cristianos, en la Edad Media.
%Es probable que las primeras acuñaciones castellanas y aragonesas nacieran ante la negativa de los reinos de taifas a pagar parias.
%Desde el siglo IX al XI surgen en Europa las acuñaciones feudales al margen de la real, tanto de la nobleza eclesiástica como de la seglar. 
Contrario a la gran diversidad de francia y alemania de monedas fe, En los reinos cristiano hispánicos, por el contrario, no existieron monedas feudales y los reyes mantuvieron el derecho exclusivo de acuñar moneda


It’s genuinely really hard and probably inadvisable to take the social out of money, not least because it’s the social risk-sharing aspect of money which facilitates economic scaling, efficiency and smoothing in the first place.






\section{Discussion}
\label{se:dis}


\bibliographystyle{apalike}
\bibliography{C:/workspace/github/bibliography-jgr/bibliojgr}

\newpage
\section*{Supporting Information}
\label{se:suppinf}

